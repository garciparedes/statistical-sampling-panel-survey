% !TEX root = ./article.tex

\documentclass{article}

\usepackage{mystyle}
\usepackage{myvars}

%-----------------------------

\begin{document}

	\maketitle

%-----------------------------
%	TEXT
%-----------------------------


	\section{Descripción}
	\label{sec:description}

    \paragraph{}
    La \emph{encuesta panel} consiste en un tipo de muestreo estadístico basado en la recolección de mediciones siguiendo una determinada línea temporal. De esta manera se obtiene un histórico acerca de la variación del objeto sobre el que se está realizando el experimento respecto del tiempo.

    \paragraph{}
    Existen dos tipos de encuestas panel, cuando todas las mediciones se hacen siguiendo un determinado patrón temporal entonces se dice que estás son \emph{balanceadas}, mientras que cuando no se sigue un patrón prefijado se habla de \emph{no balanceadas}.

%-----------------------------
%	Bibliographic references
%-----------------------------
	\nocite{muest2016}

  \bibliographystyle{alpha}
  \bibliography{bib}

\end{document}
